\documentclass[dvipdfmx,autodetect-engine,titlepage]{jsarticle}
\usepackage[dvipdfm]{graphicx}
\usepackage{ascmac}
\usepackage{fancybox}
\usepackage{listings}
\usepackage{plistings}
\usepackage{itembkbx}
\usepackage{amsmath}
\usepackage{svg}
\usepackage{url}
\usepackage{graphics}
\usepackage{listings,jvlisting}

\textheight=23cm
\renewcommand{\figurename}{図}
\renewcommand{\tablename}{表}
\newenvironment{code}
{\vspace{0.5zw}\VerbatimEnvironment  
\begin{screen} 
\baselineskip=1.0\normalbaselineskip
 \begin{Verbatim}}
{\end{Verbatim}
\baselineskip=\normalbaselineskip
 \end{screen}\vspace{0.5zw}} 

\title{情報理工学部 SNコース 3回\\
ワイヤレス通信システム\\
第一回授業(4/11)ノート}
\author{2600200443-6\\Yamashita Kyohei\\山下 恭平}
\date{Dec 7 2021}

\begin{document}

\maketitle

\section{授業内容}
・この授業の教科書はセキュリティ・ネットワーク学実験3の物と同じである。\\

・電波は電磁波の一種である、電波法により3000000MHz以下の電磁波を
「電波」と定義している。\\

・電磁波は横波である。\\

・場の理論・・・物理量が空間的に分布するとき、場を作っている。\\

・物理量にはベクトルとスカラーがあり、電磁波はベクトルである。\\

・電波は周波数を適切に設定することで、反射を調節することが可能であり、
多くの建物がある場所などでも電波が届くのはこのためである。\\

・物理的な普遍量は光速cであり、電波は光であるので、電波の周波数か波長が
分かれば、もう一方は自動的に求めることが可能である。\begin{math}
  c = \lambda f
\end{math}\\


\end{document}

