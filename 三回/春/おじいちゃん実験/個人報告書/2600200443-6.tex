\documentclass[dvipdfmx,autodetect-engine]{jsarticle}
\usepackage[dvipdfm]{graphicx}
\usepackage{ascmac}
\usepackage{fancybox}
\usepackage{listings}
\usepackage{plistings}
\usepackage{itembkbx}
\usepackage{amsmath}
\usepackage{svg}
\usepackage{url}
\usepackage{graphics}
\usepackage{listings,jvlisting}
\usepackage{here}

\lstset{
  basicstyle={\ttfamily},
  identifierstyle={\small},
  commentstyle={\smallitshape},
  keywordstyle={\small\bfseries},
  ndkeywordstyle={\small},
  stringstyle={\small\ttfamily},
  frame={tb},
  breaklines=true,
  columns=[l]{fullflexible},
  numbers=left,
  xrightmargin=0zw,
  xleftmargin=3zw,
  numberstyle={\scriptsize},
  stepnumber=1,
  numbersep=1zw,
  lineskip=-0.5ex
}

\textheight=23cm
\renewcommand{\figurename}{図}
\renewcommand{\tablename}{表}
\newenvironment{code}
{\vspace{0.5zw}\VerbatimEnvironment  
\begin{screen} 
\baselineskip=1.0\normalbaselineskip
 \begin{Verbatim}}
{\end{Verbatim}
\baselineskip=\normalbaselineskip
 \end{screen}\vspace{0.5zw}} 

 \title{10班 分担担当および成果調書} 
 \author{山下恭平}
 \date{2022年7月17日}
 \begin{document} 
 \maketitle

\section{担当した部分}
私が担当した部分は以下のものである。

\begin{quote}
  \begin{itemize}
   \item アンテナの作成
   \item 作成したアンテナの性能測定
   \item 得られた結果に対する考察
  \end{itemize}
 \end{quote}

 私は主にこの3つを担当した。全体を100\%としたとき私の占める割合は
 \textbf{33\%}である。\\

 アンテナの作成においては、チームで考えたアンテナのモデリングを行い、良い結果
 が得られたものを銅線を用いて実際に作る役割を担った。アンテナの性能評価は
 、テレビの部屋にチームメンバーと足を運び、アンテナを窓の外に固定する役割のこと
 である。この役割が意外にも計測結果に大きく影響していたので、楽しみながらではあるが
 、とてもやりがいを感じた。最後に考察だが、グループ内で議論を重ねた結果
 、最終的に私が考えた考察を全体発表に採用することとした。

 \section*{成果/感想}
 実験のコンセプトである「自分で設計した世界で一つだけのアンテナを作ろう」は
 果たすことができたと感じている。ダイポールアンテナ、八木宇田アンテナの二つを
 作り上げたが、その双方どちらも作る過程がどれも初めての経験でありとても勉強に
 なった。普段パソコンに向かいっきりの時間が多いSNコースの私にとって、実際に
 アンテナを作る経験はとても貴重なものであった。この実験での反省点を上げるなら、
 私がアンテナを作ることに偏ってしまい、モデリングを少し疎かにしてしまったことが
 挙げられる。しかし、チームメンバーの二人とうまくコミュニケーションが取れていた
 こともあり、暴走を防いでもらうことができた。作成したアンテナは決して成功
 といえるものではなかったが、得られたものはとても多いので、結果として実験には
 満足することができた。

\end{document}

