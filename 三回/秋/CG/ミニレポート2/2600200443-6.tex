\documentclass[dvipdfmx,autodetect-engine,titlepage]{jsarticle}
\usepackage[dvipdfm]{graphicx}
\usepackage{ascmac}
\usepackage{fancybox}
\usepackage{listings}
\usepackage{plistings}
\usepackage{itembkbx}
\usepackage{amsmath}
\usepackage{amssymb}
\usepackage{amsfonts}
\usepackage{svg}
\usepackage{url}
\usepackage{graphics}
\usepackage{multirow}
\usepackage{listings,jvlisting}

\textheight=23cm
\renewcommand{\figurename}{図}
\renewcommand{\tablename}{表}
\newenvironment{code}
{\vspace{0.5zw}\VerbatimEnvironment  
\begin{screen} 
\baselineskip=1.0\normalbaselineskip
 \begin{Verbatim}}
{\end{Verbatim}
\baselineskip=\normalbaselineskip
 \end{screen}\vspace{0.5zw}} 

\title{情報理工学部 SNコース 3回\\
第二回ミニレポート\\}
\author{2600200443-6\\Yamashita Kyohei\\山下 恭平}
\date{Oct 28 2022}

\begin{document}

\maketitle

\section*{問題}
視点座標から2次元スクリーン座標へのx座標の変換式で,簡単のため,d=1の時,
2点間のスクリーン座標での距離を求めて、表にまとめよ
\begin{table}[h]
  \centering
  \begin{tabular}{|c|c|c|c|}
  \hline
  z\textbackslash{} & (\begin{math}xs_1 , ys_1\end{math}) & (\begin{math}xs_2 , ys_2\end{math}) & 2点の距離   \\ \hline
  2                 & (1/2 , 0)       & (1 , 0)         & 1/2  \\ \hline
  4                 & (1/4 , 0)       & (1/2 , 0)       & 1/4  \\ \hline
  6                 & (1/6 , 0)       & (1/3 , 0)       & 1/6  \\ \hline
  8                 & (1/8 , 0)       & (1/4 , 0)       & 1/8  \\ \hline
  10                & (1/10 , 0)      & (1/5 , 0)       & 1/10 \\ \hline
  \end{tabular}
  \end{table}



\end{document}