\documentclass[dvipdfmx,autodetect-engine,titlepage]{jsarticle}
\usepackage[dvipdfm]{graphicx}
\usepackage{ascmac}
\usepackage{fancybox}
\usepackage{listings}
\usepackage{plistings}
\usepackage{itembkbx}
\usepackage{amsmath}
\usepackage{amssymb}
\usepackage{amsfonts}
\usepackage{svg}
\usepackage{url}
\usepackage{graphics}
\usepackage{multirow}
\usepackage{listings,jvlisting}

\textheight=23cm
\renewcommand{\figurename}{図}
\renewcommand{\tablename}{表}
\newenvironment{code}
{\vspace{0.5zw}\VerbatimEnvironment  
\begin{screen} 
\baselineskip=1.0\normalbaselineskip
 \begin{Verbatim}}
{\end{Verbatim}
\baselineskip=\normalbaselineskip
 \end{screen}\vspace{0.5zw}} 

\title{情報理工学部 SNコース 3回\\
第6回ミニレポート\\}
\author{2600200443-6\\Yamashita Kyohei\\山下 恭平}
\date{Dec 22 2022}

\begin{document}

\maketitle

\section*{問題}
白色の光源光 I = (1, 1, 1) が,入射角60度で,拡散反射係数 (0.8, 0, 0.6), 鏡面反射係数 (0.5, 0.5, 0.5)の平面に入射した. 環境光は Ia = (0.4, 0.4, 0.4)とする。以下の問に答えよ.

\subsection*{問1 光源光に由来する拡散反射光のR, G, B 成分を求めよ.}

光源光なので
\begin{align*}
  J_d^{(R)} &= 0.8 \times \cos 60^\circ = 0.4 \\
  J_d^{(G)} &= 0 \\
  J_d^{(B)} &= 0.6 \times \cos 60^\circ = 0.3 \\
\end{align*}

\subsection*{問2 環境光に由来する拡散反射光の R, G, B 成分を求めよ. (ただし,光源光と環境光で,考えている平面の拡散反射 係数は同じであるとする.)}

環境光なので
\begin{align*}
  J_d^{(R)} &= 0.8 \times 0.4 = 0.32 \\
  J_d^{(G)} &= 0 \\
  J_d^{(B)} &= 0.6 \times 0.4= 0.24 \\
\end{align*}

\subsection*{問3 鏡面反射光が最も強くなる視点位置での鏡面反射光の R, G, B成分を求めよ.}

鏡面反射光かつ、最も強くなる視点なので

\begin{align*}
  \alpha = 0 \\
  J_d^{(R)} &= 0.5 \times 1.0 \times 1.0 = 0.5 \\
  J_d^{(G)} &= 0.5 \times 1.0 \times 1.0 = 0.5 \\
  J_d^{(B)} &= 0.5 \times 1.0 \times 1.0 = 0.5 \\
\end{align*}


\end{document}