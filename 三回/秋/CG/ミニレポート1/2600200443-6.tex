\documentclass[dvipdfmx,autodetect-engine,titlepage]{jsarticle}
\usepackage[dvipdfm]{graphicx}
\usepackage{ascmac}
\usepackage{fancybox}
\usepackage{listings}
\usepackage{plistings}
\usepackage{itembkbx}
\usepackage{amsmath}
\usepackage{amssymb}
\usepackage{amsfonts}
\usepackage{svg}
\usepackage{url}
\usepackage{graphics}
\usepackage{multirow}
\usepackage{listings,jvlisting}

\textheight=23cm
\renewcommand{\figurename}{図}
\renewcommand{\tablename}{表}
\newenvironment{code}
{\vspace{0.5zw}\VerbatimEnvironment  
\begin{screen} 
\baselineskip=1.0\normalbaselineskip
 \begin{Verbatim}}
{\end{Verbatim}
\baselineskip=\normalbaselineskip
 \end{screen}\vspace{0.5zw}} 

\title{情報理工学部 SNコース 3回\\
コンピュータグラフィックス(Q)\\
ミニレポート1\\}
\author{2600200443-6\\Yamashita Kyohei\\山下 恭平}
\date{Oct 14 2022}

\begin{document}

\maketitle

\subsection*{問1 考えてみよう(4)}

長方形の面積を求めれば良いので.

\begin{align}
  \begin{bmatrix}
    2 & 0 & 0 & 0 \\
    0 & 2 & 0 & 0 \\
    2 & 0 & 2 & 10 \\
    2 & 0 & 0 & 1 \\ 
  \end{bmatrix}
  \begin{bmatrix}
    1 \\
    0 \\
    0 \\
    1 \\
  \end{bmatrix}
  =
  \begin{bmatrix}
    2 \\
    0 \\
    10 \\
    1 \\
  \end{bmatrix}
\end{align}

\subsection*{問2 考えてみよう(5)}


\begin{align}
  \begin{bmatrix}
    2 & 0 & 0 & 0 \\
    0 & 2 & 0 & 0 \\
    2 & 0 & 2 & 20 \\
    2 & 0 & 0 & 1 \\ 
  \end{bmatrix}
  \begin{bmatrix}
    1 \\
    0 \\
    0 \\
    1 \\
  \end{bmatrix}
  =
  \begin{bmatrix}
    2 \\
    0 \\
    20 \\
    1 \\
  \end{bmatrix}
\end{align}


\end{document}