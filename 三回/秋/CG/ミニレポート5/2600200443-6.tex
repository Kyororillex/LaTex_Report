\documentclass[dvipdfmx,autodetect-engine,titlepage]{jsarticle}
\usepackage[dvipdfm]{graphicx}
\usepackage{ascmac}
\usepackage{fancybox}
\usepackage{listings}
\usepackage{plistings}
\usepackage{itembkbx}
\usepackage{amsmath}
\usepackage{amssymb}
\usepackage{amsfonts}
\usepackage{svg}
\usepackage{url}
\usepackage{graphics}
\usepackage{multirow}
\usepackage{listings,jvlisting}

\textheight=23cm
\renewcommand{\figurename}{図}
\renewcommand{\tablename}{表}
\newenvironment{code}
{\vspace{0.5zw}\VerbatimEnvironment  
\begin{screen} 
\baselineskip=1.0\normalbaselineskip
 \begin{Verbatim}}
{\end{Verbatim}
\baselineskip=\normalbaselineskip
 \end{screen}\vspace{0.5zw}} 

\title{情報理工学部 SNコース 3回\\
第5回ミニレポート\\}
\author{2600200443-6\\Yamashita Kyohei\\山下 恭平}
\date{Dec 8 2022}

\begin{document}

\maketitle

\section*{問題}
半径 r, 中心の位置ベクトルCの球とレイの交点の求め方を,
数式を使って説明せよ

\section*{回答}

九の方程式と、直線の式を連立すると

\begin{align*}
  | \mathbf{P} -\mathbf{C}  \vert ^2 &= r^2\\
  (\mathbf{P} -\mathbf{C} )\cdot (\mathbf{P} -\mathbf{C} ) &= r^2 \\
  (\mathbf{P} _E + t\mathbf{e} - \mathbf{C}) \cdot (\mathbf{P} _E + t\mathbf{e} - \mathbf{C}) &= r^2\\
  t^2 + 2\mathbf{e}\cdot(\mathbf{P}_E - \mathbf{C})t + (\mathbf{P}_E - \mathbf{C})\cdot(\mathbf{P}_E - \mathbf{C}) - r^2 &= 0\\
\end{align*}

ここで\begin{math}b = \mathbf{e}\cdot(\mathbf{P}_E - \mathbf{C}) , c = (\mathbf{P}_E - \mathbf{C})\cdot(\mathbf{P}_E - \mathbf{C}) - r^2\end{math}とすると

\begin{align*}
  t^2 + 2bt + c &= 0\\
  t = -b \pm \sqrt{b^2 - c}
\end{align*}

よって、\begin{math}b^2>c\end{math}の時交点\begin{math}t\end{math}は

\begin{align*}
  交点t = -b \pm \sqrt{b^2 - c}\\
\end{align*}

となる
\end{document}