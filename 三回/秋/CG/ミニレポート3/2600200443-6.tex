\documentclass[dvipdfmx,autodetect-engine,titlepage]{jsarticle}
\usepackage[dvipdfm]{graphicx}
\usepackage{ascmac}
\usepackage{fancybox}
\usepackage{listings}
\usepackage{plistings}
\usepackage{itembkbx}
\usepackage{amsmath}
\usepackage{amssymb}
\usepackage{amsfonts}
\usepackage{svg}
\usepackage{url}
\usepackage{graphics}
\usepackage{multirow}
\usepackage{listings,jvlisting}

\textheight=23cm
\renewcommand{\figurename}{図}
\renewcommand{\tablename}{表}
\newenvironment{code}
{\vspace{0.5zw}\VerbatimEnvironment  
\begin{screen} 
\baselineskip=1.0\normalbaselineskip
 \begin{Verbatim}}
{\end{Verbatim}
\baselineskip=\normalbaselineskip
 \end{screen}\vspace{0.5zw}} 

\title{情報理工学部 SNコース 3回\\
第3回ミニレポート\\}
\author{2600200443-6\\Yamashita Kyohei\\山下 恭平}
\date{Nov 10 2022}

\begin{document}

\maketitle

\section*{問題}
xy 平面 (平面z=0)上にある正三角形ABC の頂点座標が,それぞれ,\begin{math}
A(0,0,0)\,B(1,0,0)\,C(\frac{1}{2},\frac{\sqrt{3}}{2},0)\end{math}
であるとする.位置ベクトルの平均として重心\begin{math}G(x_{G},y_{G},z_{G})\end{math}
を求め,それが中学/高校で習った重心と一致することを確かめよ.

\section*{回答}
重心Gは
\begin{align}
  x_{G} &= \frac{0 + 1 + \frac{1}{2}}{3} = \frac{1}{2}\notag\\
  y_{G} &= \frac{0 + 0 + \frac{\sqrt{3}}{2}}{3} = \frac{\sqrt{3}}{6}\notag\\
  z_{G} &= 0\notag\\\notag\\
  G &= (\frac{1}{2},\frac{\sqrt{3}}{6},0)
\end{align}

ここで,点Cからx軸に向かって下ろした垂線の足を点Dとすると点Dの座標は
\begin{math}D(0,\frac{1}{2},0)\end{math}となる.この時,線分CDは
三角形ABCの中線であるので,これを2:1に内分する点をEとすると
\begin{align}
  E &= (\frac{\frac{1}{2} + 2\times\frac{1}{2}}{2+1} , \frac{\frac{\sqrt{3}}{2}}{2+1})\notag\\
    &= (\frac{1}{2} , \frac{\sqrt{3}}{6})
\end{align}

(2) = (1) より,重心は一致.

\end{document}