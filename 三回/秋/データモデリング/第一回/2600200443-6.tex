\documentclass[dvipdfmx,autodetect-engine,titlepage]{jsarticle}
\usepackage[dvipdfm]{graphicx}
\usepackage{ascmac}
\usepackage{fancybox}
\usepackage{listings}
\usepackage{plistings}
\usepackage{itembkbx}
\usepackage{amsmath}
\usepackage{amssymb}
\usepackage{amsfonts}
\usepackage{svg}
\usepackage{url}
\usepackage{graphics}
\usepackage{multirow}
\usepackage{listings,jvlisting}

\textheight=23cm
\renewcommand{\figurename}{図}
\renewcommand{\tablename}{表}
\newenvironment{code}
{\vspace{0.5zw}\VerbatimEnvironment  
\begin{screen} 
\baselineskip=1.0\normalbaselineskip
 \begin{Verbatim}}
{\end{Verbatim}
\baselineskip=\normalbaselineskip
 \end{screen}\vspace{0.5zw}} 

\title{情報理工学部 SNコース 3回\\
第一回レポート(回帰分析)\\}
\author{2600200443-6\\Yamashita Kyohei\\山下 恭平}
\date{Oct 16 2022}

\begin{document}

\maketitle

\section*{問題}

2つの説明変数 x1, x2 から目的変数 y を予測することを考える

\subsection*{問1 分散,共分散を指定された表に基づき求めよ}

表より

\begin{align*}
  \overline{x_1} &= \frac{1}{10}(1.2+1.6+3.5+4+5.6+5.7+6.7+7.5+8.5+9.7) = 5.4\\\\
  \overline{x_{1}^2} &= \frac{1}{10}(1.2^2+1.6^2+3.5^2+4^2+5.6^2+5.7^2+6.7^2+7.5^2+8.5^2+9.7^2) = 36.358\\\\
  \overline{x_2} &= \frac{1}{10}(1.9+2.7+3.7+3.1+3.5+7.5+1.2+3.7+0.6+5.1) = 3.3\\\\
  \overline{x_{2}^2} &= \frac{1}{10}(1.9^2+2.7^2+3.7^2+3.1^2+3.5^2+7.5^2+1.2^2+3.7^2+0.6^2+5.1^2) = 14.42\\\\
  \overline{y} &= \frac{1}{10}(0.9+1.3+2+1.8+2.2+3.5+1.9+2.7+2.1+3.6) = 2.2\\\\
  \overline{x_{1}x_2} &= \frac{1}{10} \sum_{n = 1}^{10} x_{1_n}x_{2_n} = 18.538\\\\
  \overline{x_{1}y} &= \frac{1}{10} \sum_{n = 1}^{10} x_{1_n}y_n = 13.722\\\\
  \overline{x_{2}y} &= \frac{1}{10} \sum_{n = 1}^{10} x_{2_n}y_n = 8.531\\\\
\end{align*}

よって


\begin{align*}
  s_{x_1}^2 &= \overline{x_{1}^2} - \overline{x_1}^2 = 7.198 \\\\
  s_{x_2}^2 &= \overline{x_{2}^2} - \overline{x_2}^2 = 3.53 \\\\
  s_{x_{1}x_{2}} &=  \overline{x_{1}x_2} - \overline{x_1}\,\overline{x_2} = 0.718 \\\\
  s_{x_{1}y} &=  \overline{x_{1}y} - \overline{x_1}\,\overline{y} = 1.842 \\\\
  s_{x_{2}y} &=  \overline{x_{2}y} - \overline{x_2}\,\overline{y} = 1.271
\end{align*}

\subsection*{問2 分散共分散行列を求めよ}

問1より

\begin{align*}
  \begin{pmatrix}
    s_{x_1}^2 & s_{x_{1}x_{2}} \\
    s_{x_{1}x_{2}} & s_{x_2}^2 
  \end{pmatrix}
    &=
  \begin{pmatrix}
    7.198 & 0.718 \\
    0.718 & 3.53 
  \end{pmatrix}
\end{align*}


\subsection*{問3 回帰方程式を求めよ}

\begin{align*}
  \overline{y} &=  a_0 + a_{1}\overline{x_1} + a_{2}\overline{x_2}
\end{align*}

であるので,問1より

\begin{align}
  2.2 &= a_0 + 5.4a_1 + 3.3a_2
\end{align}

この時

\begin{align*}
  \begin{pmatrix}
    s_{x_1}^2 & s_{x_{1}x_{2}} \\
    s_{x_{1}x_{2}} & s_{x_2}^2 
  \end{pmatrix}
  \begin{pmatrix}
    a_1\\
    a_2
  \end{pmatrix}
  &=
  \begin{pmatrix}
    s_{x_{1}y}\\
    s_{x_{2}y}
  \end{pmatrix}
  \\\\
  \begin{pmatrix}
    7.198 & 0.718 \\
    0.718 & 3.53 
  \end{pmatrix}
  \begin{pmatrix}
    a_1\\
    a_2
  \end{pmatrix}
  &=
  \begin{pmatrix}
    1.842\\
    1.271
  \end{pmatrix}
\end{align*}

となるので、クラメルの公式より

\begin{align*}
  a_1 &= \frac{\begin{vmatrix}1.842 & 0.718 \\1.271 & 3.53 \end{vmatrix}}{\begin{vmatrix}7.198 & 0.718 \\ 0.718 & 3.53 \end{vmatrix}} = 0.2245 \\\\
  a_2 &= \frac{\begin{vmatrix}7.198 & 1.842 \\0.718 & 1.271 \end{vmatrix}}{\begin{vmatrix}7.198 & 0.718 \\ 0.718 & 3.53 \end{vmatrix}} = 0.3144
\end{align*}

式(1)に代入すると

\begin{align*}
  2.2 &= a_0 + 5.4 \times 0.2245 + 3.3 \times 0.3144 \\
  a_0 &= -0.04982
\end{align*}

したがって求める回帰方程式は

\begin{align*}
  y = 0.2245x_1 + 0.3144x_2 - 0.04982
\end{align*}



\end{document}