\documentclass[dvipdfmx,autodetect-engine,titlepage]{jsarticle}
\usepackage[dvipdfm]{graphicx}
\usepackage{ascmac}
\usepackage{fancybox}
\usepackage{listings}
\usepackage{plistings}
\usepackage{itembkbx}
\usepackage{amsmath}
\usepackage{amssymb}
\usepackage{amsfonts}
\usepackage{svg}
\usepackage{url}
\usepackage{graphics}
\usepackage{multirow}
\usepackage{listings,jvlisting}

\textheight=23cm
\renewcommand{\figurename}{図}
\renewcommand{\tablename}{表}
\newenvironment{code}
{\vspace{0.5zw}\VerbatimEnvironment  
\begin{screen} 
\baselineskip=1.0\normalbaselineskip
 \begin{Verbatim}}
{\end{Verbatim}
\baselineskip=\normalbaselineskip
 \end{screen}\vspace{0.5zw}} 

\title{情報理工学部 SNコース 3回\\
第二回レポート(固有値,固有ベクトル)\\}
\author{2600200443-6\\Yamashita Kyohei\\山下 恭平}
\date{Oct 23 2022}

\begin{document}

\maketitle

\section*{問題}

\begin{align*}
  A
    &=
  \begin{pmatrix}
    1 & 2 \\
    2 & -2 
  \end{pmatrix}
\end{align*}

の時以下の問いに答えよ

\subsection*{問1 対称行列Aの固有値\begin{math}\alpha,\beta \end{math}を求めよ}

\begin{align*}
 det
  \begin{pmatrix}
    1-\lambda & 2 \\
    2 & -2-\lambda 
  \end{pmatrix}
  &= 0 \\
  (1-\lambda)(-2-\lambda)-4 &= 0\\
  \lambda^2 + \lambda -6 &= 0 \\
  (\lambda+3)(\lambda-2) &= 0\\
  \lambda = 2,-3
\end{align*}
よって
\begin{align*}
  \alpha,\beta &= 2 , -3
 \end{align*}

\subsection*{問2 固有値に対応する,大きさ1のベクトルを求めよ}

\begin{math}
  \lambda = \alpha = 2
\end{math} 
の時,固有ベクトルは
\begin{align*}
  \begin{pmatrix}
    -1 & 2 \\
    2 & -4
  \end{pmatrix}
  \begin{pmatrix}
    x_1 \\
    x_2
  \end{pmatrix}
  =
  \overrightarrow{0}\\\\
  x_1 = 2x_2
\end{align*}
となるので,大きさ1の固有ベクトルは
\begin{align}
  \overrightarrow{e_1} =   \begin{pmatrix}
    \frac{2\sqrt{5} }{5} \\
    \frac{\sqrt{5}}{5}
  \end{pmatrix}
\end{align}

\begin{math}
  \lambda = \beta = -3
\end{math} 
の時,固有ベクトルは
\begin{align*}
  \begin{pmatrix}
    4 & 2 \\
    2 & 1
  \end{pmatrix}
  \begin{pmatrix}
    x_1 \\
    x_2
  \end{pmatrix}
  =
  \overrightarrow{0}\\\\
  x_1 = -\frac{1}{2}x_2
\end{align*}
となるので,大きさ1の固有ベクトルは
\begin{align}
  \overrightarrow{e_2} =   \begin{pmatrix}
    \frac{\sqrt{5}}{5} \\
    -\frac{2\sqrt{5} }{5}
  \end{pmatrix}
\end{align}

\subsection*{問3 固有ベクトルが直行しているこを示せ}

\begin{align*}
  \overrightarrow{e_1}\cdot \overrightarrow{e_2} &= \frac{2\sqrt{5} }{5}\cdot \frac{\sqrt{5}}{5} - \frac{\sqrt{5}}{5} \cdot \frac{2\sqrt{5} }{5}\\
  &=0
\end{align*}

よって直行

\subsection*{問4 大きさ1の固有ベクトル正規直交行列をPとする.\\\\
\begin{math}P^{-1}AP =  \begin{pmatrix}
  \alpha  & 0 \\
  0 & \beta  
\end{pmatrix}\end{math}となることを示せ}

\begin{align*}
  P =   \begin{pmatrix}
    \frac{2\sqrt{5}}{5} & \frac{\sqrt{5}}{5} \\
    \frac{\sqrt{5}}{5} & \frac{-2\sqrt{5}}{5}
  \end{pmatrix}
\end{align*}
この時
\begin{align*}
  P^{-1} &=   
  \begin{pmatrix}
    \frac{2\sqrt{5}}{5} & \frac{\sqrt{5}}{5} \\
    \frac{\sqrt{5}}{5} & \frac{-2\sqrt{5}}{5}
  \end{pmatrix}
\end{align*}

よって

\begin{align}
  P^{-1}AP &=  
  \begin{pmatrix}
    \alpha  & 0 \\
    0 & \beta  
  \end{pmatrix}\notag\\
  PP^{-1}APP^{-1} &= P
  \begin{pmatrix}
    2  & 0 \\
    0 & -3  
  \end{pmatrix}
  P^{-1}\notag\\
  A &= 
  \begin{pmatrix}
    \frac{2\sqrt{5}}{5} & \frac{\sqrt{5}}{5} \\
    \frac{\sqrt{5}}{5} & \frac{-2\sqrt{5}}{5}
  \end{pmatrix}
  \begin{pmatrix}
    2  & 0 \\
    0 & -3  
  \end{pmatrix}
  \begin{pmatrix}
    \frac{2\sqrt{5}}{5} & \frac{\sqrt{5}}{5} \\
    \frac{\sqrt{5}}{5} & \frac{-2\sqrt{5}}{5}
  \end{pmatrix}\notag\\
  &=
  \begin{pmatrix}
    \frac{4\sqrt{5}}{5} & \frac{-3\sqrt{5}}{5} \\
    \frac{2\sqrt{5}}{5} & \frac{6\sqrt{5}}{5}
  \end{pmatrix}
  \begin{pmatrix}
    \frac{2\sqrt{5}}{5} & \frac{\sqrt{5}}{5} \\
    \frac{\sqrt{5}}{5} & \frac{-2\sqrt{5}}{5}
  \end{pmatrix}\notag\\
  &=
  \begin{pmatrix}
    1 & 2 \\
    2 & -2 
  \end{pmatrix}
  =(左辺)
\end{align}

題意は示された

\end{document}