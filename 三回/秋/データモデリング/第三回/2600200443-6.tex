\documentclass[dvipdfmx,autodetect-engine,titlepage]{jsarticle}
\usepackage[dvipdfm]{graphicx}
\usepackage{ascmac}
\usepackage{fancybox}
\usepackage{listings}
\usepackage{plistings}
\usepackage{itembkbx}
\usepackage{amsmath}
\usepackage{amssymb}
\usepackage{amsfonts}
\usepackage{svg}
\usepackage{url}
\usepackage{graphics}
\usepackage{multirow}
\usepackage{listings,jvlisting}

\textheight=23cm
\renewcommand{\figurename}{図}
\renewcommand{\tablename}{表}
\newenvironment{code}
{\vspace{0.5zw}\VerbatimEnvironment  
\begin{screen} 
\baselineskip=1.0\normalbaselineskip
 \begin{Verbatim}}
{\end{Verbatim}
\baselineskip=\normalbaselineskip
 \end{screen}\vspace{0.5zw}} 

\title{情報理工学部 SNコース 3回\\
第3回レポート(SVM)\\}
\author{2600200443-6\\Yamashita Kyohei\\山下 恭平}
\date{Nov 18 2022}

\begin{document}

\maketitle

\section*{問題}

\begin{align*}
  \overrightarrow{x}_1 
    =
  \begin{pmatrix}
    0 \\
    1  
  \end{pmatrix}
  ,
  \overrightarrow{x}_2 
    =
  \begin{pmatrix}
    1 \\
    1  
  \end{pmatrix}
  ,
  D = \left\{(\overrightarrow{x}_1,-1) , (\overrightarrow{x}_2 , 1)\right\} 
\end{align*}
の時以下の問いに答えよ

\subsection*{問1 以下の内積を計算せよ}

\begin{align*}
  (\overrightarrow{x}_1,\overrightarrow{x}_1) &= 1\\
  (\overrightarrow{x}_1,\overrightarrow{x}_2) &= 1\\
  (\overrightarrow{x}_2,\overrightarrow{x}_1) &= 1\\
  (\overrightarrow{x}_2,\overrightarrow{x}_2) &= 2
\end{align*}


\subsection*{問2 未定乗数\begin{math}\alpha_1,\alpha_2\end{math}を使って,ラグランジュ関数を書け}

\begin{align*}
  L_D &= \sum_{i = 1}^{2} \alpha_i - \frac{1}{2}\sum_{i = 1}^{2} \sum_{k = 1}^{2} \alpha_i \alpha_k y_i y_k(\overrightarrow{x}_i , \overrightarrow{x}_k)\\
      &= (\alpha_1 + \alpha_2) - \frac{1}{2}\left\{\alpha_1 \alpha_1 y_1 y_1 (\overrightarrow{x}_1 , \overrightarrow{x}_1) + \alpha_1 \alpha_2 y_1 y_2 (\overrightarrow{x}_1 , \overrightarrow{x}_2)
        + \alpha_2 \alpha_1 y_2 y_1 (\overrightarrow{x}_2 , \overrightarrow{x}_1) + \alpha_2 \alpha_2 y_2 y_2 (\overrightarrow{x}_2 , \overrightarrow{x}_2)\right\} \\
      &= (\alpha_1 + \alpha_2) - \frac{1}{2}(\alpha_1^2 y_1^2 + 2\alpha_1 \alpha_2 y_1 y_2 + 2\alpha_2^2 y_2^2)
\end{align*}

\subsection*{問3 KKT 条件のうちの式 (5) をつかって,未定定数の関係を示せ}

\begin{align*}
  \sum_{i = 1}^{\infty} \alpha_i y_i &= 0\\
\end{align*}

より

\begin{align*}
  \alpha_1 y_1 &+ \alpha_2 y_2 = 0\\
  \alpha_2 &= -\frac{y_1}{y_2}\alpha_1
\end{align*}

\subsection*{問4 問2と問3の結果から,\begin{math}\alpha_2\end{math}を消去し,簡潔化されたラグランジュ関数を書け}

問2,問3より

\begin{align*}
  L_D = (1 - \frac{y_1}{y_2})\alpha_1 - \frac{1}{2}\alpha_1^2 y_1^2
\end{align*}

\subsection*{問5 偏微分の結果が0であることを用いて,\begin{math}\alpha_1,\alpha_2\end{math}を求めよ}

\begin{align*}
  \frac{\partial}{\partial \alpha_1}\{ (1 - \frac{y_1}{y_2})\alpha_1 - \frac{1}{2}\alpha_1^2 y_1^2 \}  &= 0\\
  -\alpha_1 y_1^2 - \frac{y_1}{y_2} + 1 = 0\\
  \alpha_1 = \frac{1}{y_1^2} - \frac{1}{y_1 y_2}
\end{align*}

よって

\begin{align*}
  \alpha_2 = -\frac{1}{y_2^2} + \frac{1}{y_1 y_2}
\end{align*}

\subsection*{問6 式(2)より,\begin{math}\overrightarrow{w}^*\end{math}を求めよ}

式(2)より

\begin{align*}
  \overrightarrow{w} &= \alpha_1 y_1 \overrightarrow{x}_1 + \alpha_2 y_2 \overrightarrow{x}_2\\
  &= (\frac{1}{y_1^2} - \frac{1}{y_1 y_2}) y_1   
  \begin{pmatrix}
    0 \\
    1  
  \end{pmatrix}
  +
  (-\frac{1}{y_2^2} + \frac{1}{y_1 y_2}) y_2 
  \begin{pmatrix}
    1 \\
    1  
  \end{pmatrix}\\
  &= 
  \begin{pmatrix}
    \frac{1}{y_1} - \frac{1}{y_2} \\\\
    \frac{2}{y_1} - \frac{2}{y_2}
  \end{pmatrix}
\end{align*}

\subsection*{問7 式(3)より,bを求めよ}


\end{document}