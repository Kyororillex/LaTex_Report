\documentclass[dvipdfmx,autodetect-engine,titlepage]{jsarticle}
\usepackage[dvipdfm]{graphicx}
\usepackage{ascmac}
\usepackage{fancybox}
\usepackage{listings}
\usepackage{plistings}
\usepackage{itembkbx}
\usepackage{amsmath}
\usepackage{amssymb}
\usepackage{amsfonts}
\usepackage{svg}
\usepackage{url}
\usepackage{graphics}
\usepackage{multirow}
\usepackage{listings,jvlisting}

\textheight=23cm
\renewcommand{\figurename}{図}
\renewcommand{\tablename}{表}
\newenvironment{code}
{\vspace{0.5zw}\VerbatimEnvironment  
\begin{screen} 
\baselineskip=1.0\normalbaselineskip
 \begin{Verbatim}}
{\end{Verbatim}
\baselineskip=\normalbaselineskip
 \end{screen}\vspace{0.5zw}} 

\title{暗号理論(B1)\\最終レポート}
\author{山下 恭平\\学籍番号:2600200443-6}
\date{2023年1月19日}

\begin{document}

\maketitle

\section{AES(Advanced Encryption Standard)}


\subsection{共通鍵暗号/AES}

AESとは「Advanced Encryption Standard」の略であり共通鍵暗号の一種である.
AESはDaemenとRijimen の提案したRijndaelという暗号化アルゴリズムを基に開発された.

\subsection{Rijndael}

\subsection{title}

RC4:1987年に開発、アルゴリズムは非公開であったが1994年に何者かによって漏洩させられ,2015年2月
にはTLSのすべてのバージョンにおいてRC4の利用を禁止する提議 RFC 7465 \begin{math}
  ^{[1]}
\end{math}が公開された.

アメリカ国立標準技術研究所(NIST)



\section{TLS}

\subsection{TLSとは}

\subsection{title}

\subsection{title}

% 参考文献はここに記述
\begin{thebibliography}{99}
    \bibitem{NVD CVSS}A complete guide to the common vulnerability scoring system.\\
      \url{https://www.rfc-editor.org/rfc/rfc7465} . (15/12/2022)
    \bibitem{main} Octavian Suciu , Connor Nelson , Zhuoer Lyu , Tiffany Bao , and Tudor Dumitras.
    \quad Expected Exploitability: Predicting the Development of Functional Vulnerability 
    Exploits. In 31th USENIX Security Symposium (USENIX Security 22), pages 377-394, 2022
\end{thebibliography}

\end{document}