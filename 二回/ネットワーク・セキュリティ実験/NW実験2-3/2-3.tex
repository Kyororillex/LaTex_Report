\documentclass[dvipdfmx,autodetect-engine,titlepage]{jsarticle}
\usepackage[dvipdfm]{graphicx}
\usepackage{ascmac}
\usepackage{fancybox}
\usepackage{listings}
\usepackage{plistings}
\usepackage{itembkbx}
\usepackage{amsmath}
\usepackage{svg}
\usepackage{url}
\usepackage{graphics}
\usepackage{listings,jvlisting}
\usepackage{scalefnt}

\textheight=23cm
\renewcommand{\figurename}{図}
\renewcommand{\tablename}{表}
\newenvironment{code}
{\vspace{0.5zw}\VerbatimEnvironment  
\begin{screen} 
\baselineskip=1.0\normalbaselineskip
 \begin{Verbatim}}
{\end{Verbatim}
\baselineskip=\normalbaselineskip
 \end{screen}\vspace{0.5zw}} 

\title{情報理工学部 SNコース 2回\\
セキュリティ・ネットワーク学実験2\\
NW実験2-3レポート}
\author{2600200443-6\\Yamashita Kyohei\\山下 恭平}
\date{November 17 2021}

\begin{document}

\maketitle

\section{概要}
自宅のネットワーク環境、および学内ネットワーク環境それぞれで、指定された
アドレスへpingコマンドを使用し、その結果を表にまとめ、結果の違いについて
考察する。

\section{結果}

\begin{table}[h]
  \centering
  \caption{学内}
  \begin{tabular}{|c|c|c|}
  \hline
  コンピュータの種類       & ホスト名またはIPアドレス      & 応答             \\ \hline
  立命館大学 webサイト    & www.ritsumei.ac.jp & なし             \\ \hline
  情報理工学部実験用サーバ   & 172.25.11.22       & 平均RTT:5.636ms  \\ \hline
  実験室内の別PC        & 172.27.74.63       & なし             \\ \hline
  BKCのDNSサーバ     & 172.24.32.1        & 平均RTT:5.915ms  \\ \hline
  衣笠のDNSサーバ       & 133.19.222.4       & 平均RTT:5.470ms  \\ \hline
  OICのDNSサーバ      & 172.23.0.1         & 平均RTT90.403ms  \\ \hline
  立命館慶祥高校のDNSサーバ & 172.20.2.7         & 平均RTT:28.768ms \\ \hline
  \end{tabular}
  \end{table}

  \begin{table}[h]
    \centering
    \caption{自宅}
    \begin{tabular}{|c|c|c|}
    \hline
    コンピュータの種類       & ホスト名またはIPアドレス      & 応答             \\ \hline
    立命館大学 webサイト    & www.ritsumei.ac.jp & なし             \\ \hline
    情報理工学部実験用サーバ   & 172.25.11.22       & なし  \\ \hline
    実験室内の別PC        & 172.27.74.63       & なし             \\ \hline
    BKCのDNSサーバ     & 172.24.32.1        & なし  \\ \hline
    衣笠のDNSサーバ       & 133.19.222.4       & なし  \\ \hline
    OICのDNSサーバ      & 172.23.0.1         & なし  \\ \hline
    立命館慶祥高校のDNSサーバ & 172.20.2.7         & なし \\ \hline
    \end{tabular}
    \end{table}

\section{考察}
自宅でのネットワーク環境で行ったところ、全てのIPアドレスで応答せずにパケットロスが100\%と
なった。ここで、次に「google.com」へpingコマンドを実行したところ、Googleのサーバは応答し
結果を返した。よって、今回の実験で全てのサーバが応答しなかったのは、学校のDNSサーバや
実験室のサーバは学内LAN以外からのアクセスを拒否していることが考えられる。


\end{document}

