\documentclass[dvipdfmx,autodetect-engine,titlepage]{jsarticle}
\usepackage[dvipdfm]{graphicx}
\usepackage{ascmac}
\usepackage{fancybox}
\usepackage{listings}
\usepackage{plistings}
\usepackage{itembkbx}
\usepackage{amsmath}
\usepackage{svg}
\usepackage{url}
\usepackage{graphics}
\usepackage{listings,jvlisting}

\textheight=23cm
\renewcommand{\figurename}{図}
\renewcommand{\tablename}{表}
\newenvironment{code}
{\vspace{0.5zw}\VerbatimEnvironment  
\begin{screen} 
\baselineskip=1.0\normalbaselineskip
 \begin{Verbatim}}
{\end{Verbatim}
\baselineskip=\normalbaselineskip
 \end{screen}\vspace{0.5zw}} 

\title{情報理工学部 SNコース 2回\\
教養ゼミナール\\
}
\author{2600200443-6\\Yamashita Kyohei\\山下 恭平}
\date{Jan 30 2022}

\begin{document}

\maketitle

\section{授業についての感想等}
この授業の前半部分で取り組んだ文献購読では、創造的思考力の鍛え方や、育成
方法について学ぶことができた。私はグループBで「プロジェクト」について良く
調べ、資料を作成したが、その中でもたくさんのことを学ぶことができた。まず、
「メイカー」という単語を知ったのもこの時初めてで、オバマ大統領がこのことに
ついて言及していたり、世界中でメイカームーブメントが起こっていることなども
知ることができた。この様な運動が起こっていることは知らなかったが、調べていくうちに
自分自身もメイカーであることに気がつき、学生として、創造的思考を育成することが
重要であることに気がつかされた。自分の創造物は新しいテクノロジーであり、
ビジネスの可能性を秘めており、産業革命を起こす可能性まであると考えると、
とてもワクワクしたのを覚えている。また、私は情報理工学部ということもあり、
アプリケーションを友達と開発したことがあるが、何気なく開発を行っていただけ
と感じていたが、この授業の後に、アプリケーション開発中に起こった様々な
エラーの対処や、データの管理方法などについてトライアンドエラーを繰り返したのは
クリエイティブラーニングスパイラルの経験であり、物作りに対する一連のプロセス
を学べていたということに気づいた。シーモア・パパートが述べた「コンピュータは表現のための新し
い媒体であり、何かを作るための新しいツールである。」という言葉がとても印象的で、
現代ではほとんどの人がコンピュータを使用しているが、「計算」を行う目的で
コンピュータを使用する機会はほとんどないと感じたからである。当然、コンピュータ
内部では様々な演算が行われているが、私達ユーザはそれを認識することなくコンピュータ
を使用している。そして、現在の多くのインターネット上のコンテンツは「表現」に
よって成り立っている。動画や芸術、ニュースなども「表現」であると考えられるので、
コンピュータの使用目的が昔と比べて大きく変わっている様に感じられるのがとても
面白いと感じた。スクラッチプログラムが他の子供向け言語に比べて優れていることは
すぐに気づくことができた。私もプログラミングスクールの資料などを見たときに、
多くの物はただ「パズル」を解いているだけだったが、スクラッチのプロジェクトでは、
高度なコーディング技術などを必要とせず、本物のプログラムと同等のプログラムを
作成することができたからである。つまり、与えられた課題を解くためのプログラムではなく
、自分で創造を行うプログラムだったのがとても驚かされたことである。私はスクラッチ
のワークショップに参加し、子供達にスクラッチプログラムを教えたのだが、パズルを解く
ものと比べて圧倒的に教えるのが難しと感じた。子供向けとはいえ、本物さながらの
コーディングを行うので、「プログラミング的思考」をどの様にして子供達に伝えるかで
とても苦労したと同時に、自分自身の思考力も大きく向上させることができた。この経験は
後半のスクラッチプログラミングプロジェクトの作成でとても役立った。どのような
ものを作るかまでを決めた後に、比較的スムーズにプログラムを作成することができたのは
間違いなく、ワークショップを通して思考力が向上したからである。チームで分担して
コーディングを行ったが、スクラッチなら人数をかけた分担作業も意外と可能ということに
気づいた、スクラッチを使えばオブジェクト指向プログラムの基礎を学べると感じた。
チームで一つのプロジェクト、プレゼンを作ることは大学生になってほぼ初めての経験で
あったので、とても貴重な体験をすることができた。今後自分が社会に出たときに、同じ様
にチームで物を作るときのイメージを抱くことができたので、とても良かった。講義
全体を通して得られるものがとても多く、非常に満足度の高い授業であった.
\end{document}

