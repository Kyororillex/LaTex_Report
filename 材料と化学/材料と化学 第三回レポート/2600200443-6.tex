\documentclass[dvipdfmx,autodetect-engine,titlepage]{jsarticle}
\usepackage[dvipdfm]{graphicx}
\usepackage{ascmac}
\usepackage{fancybox}
\usepackage{listings}
\usepackage{plistings}
\usepackage{itembkbx}
\usepackage{amsmath}
\usepackage{svg}
\usepackage{url}
\usepackage{graphics}
\usepackage{listings,jvlisting}

\textheight=23cm
\renewcommand{\figurename}{図}
\renewcommand{\tablename}{表}
\newenvironment{code}
{\vspace{0.5zw}\VerbatimEnvironment  
\begin{screen} 
\baselineskip=1.0\normalbaselineskip
 \begin{Verbatim}}
{\end{Verbatim}
\baselineskip=\normalbaselineskip
 \end{screen}\vspace{0.5zw}} 

\title{情報理工学部 SNコース 2回\\
材料と化学\\
第三回レポート}
\author{2600200443-6\\Yamashita Kyohei\\山下 恭平}
\date{Jan 17 2022}

\begin{document}

\maketitle

\section{概要}
超高分子量ポリエチレンについての構造、特徴、合成法、用途などについてまとめ、
今後のこの材料についての応用や懸念について述べる。

\section{超高分子量ポリエチレン}

\subsection{構造}
\begin{math}
  エチレン(C_2H_4)を重合することで得られる。構造としては-(CH_2-CH_2)_n-となっており、
  通常のポリエチレンの分子量が20万から30万なのに対して、超高分子量ポリエチレン
  では100万から700万まで高めている。^{(1)}
\end{math}

\subsection{特徴}
\begin{math}
  強くて、安心安全という特徴を持つ。具体的には、高い摩耗耐性、とても低い摩擦係数
、強い衝撃耐性、割れにくい、高い非粘着性、生理的に無毒、優れた薬品耐性、ほとんど
吸水しないなどが特性として挙げられる。^{(2)}
\end{math}

\subsection{合成法}
\begin{math}
  エチレンを重合し、分子量を100万から700万まで高めることで合成可能。1931年では
高圧法、1947年には低圧法で合成され、、1952年にはチーグラー触媒により常温
常圧での合成が可能となった。^{(3)}
\end{math}

\subsection{用途}
\begin{math}
  強くて、安心安全ということもあり非常に幅広く利用されている。無毒で高い耐久性
を持つので、食品加工の機械に使用され、低い摩擦係数に衝撃耐性を備えているので
スノーモービルの部品やスキー、ボーリングのレーンにも使用される、さらに、耐水性
、薬品耐性を持っているのでケミカルポンプやガスケットとしても利用されており、
食品機械、建設機械、スポーツ用品、化学用品など非常に幅広い分野で使用されている。^{(4)}
\end{math}

\section{この材料の今後}
\begin{math}
  この材料の日本国内での生産量は2003年時点で5000トンであり、これは世界の5.6\%
を占めている。^{(5)}このデータはかなり古いものであるが、現在ではさらに使用量が増えていると考えて
良いだろう。なぜなら、CiNii Articlesで超高分子ポリエチレンを調べたところ、
最近に出版された論文がいくつか見つけることができたからである、つまりこれは、
現代でも研究開発が行われているということである。近年のこの材料の使用方法
としては、人体に無毒であることを利用して、人工関節、医療機器、によく使用
されていることがわかった。さらに、今後はロボット分野でも使用されていくと
考えられる。
\end{math}

\section{参考文献}

(1)プラスチック素材辞書\\
\url{https://www.plastics-material.com/uhpe/}\\

(2)三ツ星ベルト株式会社\\
\url{https://www.mitsuboshi.com/japan/product/synthetic/kind/uhmw.html}\\

(3)高橋 達見,超高分子量ポリエチレン樹脂の成形加工技術,1990\\
\url{https://www.jstage.jst.go.jp/article/fiber1944/47/2/47_2_P100/_pdf/-char/en}\\

(4)片山チェン株式会社\\
\url{http://www.kana.co.jp/wordpress/wp-content/uploads/2012/10/2f0c8e7ba32e921591a523a8ddb68e93.pdf}\\

(5)EnplaNet.com\\
\url{https://www.enplanet.com/Ja/Output/Data/o024.html}

\end{document}

