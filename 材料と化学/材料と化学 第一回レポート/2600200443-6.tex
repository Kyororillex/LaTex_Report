\documentclass[dvipdfmx,autodetect-engine,titlepage]{jsarticle}
\usepackage[dvipdfm]{graphicx}
\usepackage{ascmac}
\usepackage{fancybox}
\usepackage{listings}
\usepackage{plistings}
\usepackage{itembkbx}
\usepackage{amsmath}
\usepackage{svg}
\usepackage{url}
\usepackage{graphics}
\usepackage{listings,jvlisting}

\textheight=23cm
\renewcommand{\figurename}{図}
\renewcommand{\tablename}{表}
\newenvironment{code}
{\vspace{0.5zw}\VerbatimEnvironment  
\begin{screen} 
\baselineskip=1.0\normalbaselineskip
 \begin{Verbatim}}
{\end{Verbatim}
\baselineskip=\normalbaselineskip
 \end{screen}\vspace{0.5zw}} 

\title{情報理工学部 SNコース 2回\\
材料と化学\\
第一回レポート}
\author{2600200443-6\\Yamashita Kyohei\\山下 恭平}
\date{Oct 17 2021}

\begin{document}

\maketitle

\section{各元素についての調査}

\subsection{水素}
次世代エネルギーとしても注目されており、最近ではTOYOTAのCMでよく見かける\\
液体酸素と液体水素の混合燃料はロケットに使用される。

\subsection{ヘリウム}
空気よりも軽く、風船などによく使用される。\\
液体ヘリウムは4K($-269^\circ$C)とかなりの低音度であるため、多くの研究機関などで利用される。

\subsection{リチウム}
全ての金属元素の中で最も軽い。\\
電池などでよく利用される。

\subsection{炭素}
身の回りの多くの物には炭素が含まれている。主要なものとして木材、紙、ダイヤモンドなどは炭素からできていると言っても過言ではない。

\subsection{窒素}
空気中で最も多く存在する元素。\\
窒素化合物は人体には重要であり、アミノ酸アンパク質などに含まれる。

\subsection{酸素}
多くの生物にとってなくてはならない元素。\\
強い火力を得られるため、金属の切断機などにも利用されている。

\subsection{フッ素}
フッ素単体では猛毒であり、単体で利用されることは少ない。\\
フッ化水素はガラスを腐食させるためガラス細工に使用される。

\subsection{ナトリウム}
アルカリとして知られる水酸化ナトリウムで広く使用されている。

\subsection{マグネシウム}
人間の生命活動を支える必須元素の一つである。\\
軽いため、自動車の軽量化素材としてもされる。

\subsection{アルミニウム}
「電気の缶詰」とも呼ばれ、精製するのに非常に多くの電力を必要とする\\
アルミホイルやビール缶、1円硬貨などに死闘されている。

\subsection{硫黄}
ゴムの加工に使用される。

\subsection{カルシウム}
大理石、骨の主な成分。また、セメント、石膏ボード、モルタルなど建築業でも
よく使用される。

\subsection{チタン}
自動車のマフラーや自転車のフレームなどに使用される。\\
加工により様々な色にできるため、コップやスプーンなどの食器に使用される
こともある。

\subsection{クロム}
メッキに使用される。また、鉄、ニッケルの合金として有名なステンレスの生成にも
使用する。

\subsection{マンガン}
単体で利用されることはほとんどないが、二酸化マンガンは多くの電池で使用される。

\subsection{鉄}
地球上にかなり多く存在する。\\
非常に様々なものに使用され、身近なものだけでも、フライパン、工具、家具など。

\subsection{ニッケル}
メッキに使用される。また、形状記憶合金などの合金を生成するのにも使用される。

\subsection{銅}
日本でも古くから使用され、十円玉、銅鍋など現代でも多く使われている。

\subsection{亜鉛}
メッキなどに使用される。人体にも必須な元素で免疫や味覚に関する役割を持つので、
コロナウイルスの時に少し注目されていた。

\subsection{ガリウム}
多くの半導体に利用される。\\
また融点が室温より少し高いので、スプーン曲げ用のスプーンなどにも利用される。

\section{感想等}
与えられた周期表を読み、非常に驚かされた。なぜなら、ほとんど全ての元素が
あらゆることに利用されていたからである。聞いたこともないような元素が、日常に
多く潜んでいることは非常に面白いと感じた。私が書いた20個も、私自身はそれなりに
何に利用されているかを把握しているつもりであったが、私が把握しているのはほんの
一部であることがわかった。このレポートを通して、日常に潜む化学により興味を持つ
ことができた。


\section{参考文献等}
wikipedia\\

\url{https://ja.wikipedia.org/wiki/液体ヘリウム}\\

\url{https://ja.wikipedia.org/wiki/リチウム}\\

\url{https://ja.wikipedia.org/wiki/フッ素}\\

\url{https://ja.wikipedia.org/wiki/マグネシウム}\\

\url{https://ja.wikipedia.org/wiki/マンガン}\\

\url{https://ja.wikipedia.org/wiki/ガリウム}\\


\end{document}

