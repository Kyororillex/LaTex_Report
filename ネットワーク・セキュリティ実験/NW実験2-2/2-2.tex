\documentclass[dvipdfmx,autodetect-engine,titlepage]{jsarticle}
\usepackage[dvipdfm]{graphicx}
\usepackage{ascmac}
\usepackage{fancybox}
\usepackage{listings}
\usepackage{plistings}
\usepackage{itembkbx}
\usepackage{amsmath}
\usepackage{svg}
\usepackage{url}
\usepackage{graphics}
\usepackage{listings,jvlisting}
\usepackage{scalefnt}

\textheight=23cm
\renewcommand{\figurename}{図}
\renewcommand{\tablename}{表}
\newenvironment{code}
{\vspace{0.5zw}\VerbatimEnvironment  
\begin{screen} 
\baselineskip=1.0\normalbaselineskip
 \begin{Verbatim}}
{\end{Verbatim}
\baselineskip=\normalbaselineskip
 \end{screen}\vspace{0.5zw}} 

\title{情報理工学部 SNコース 2回\\
セキュリティ・ネットワーク学実験2\\
NW実験2-2レポート}
\author{2600200443-6\\Yamashita Kyohei\\山下 恭平}
\date{November 17 2021}
\begin{document}

\maketitle

\section{概要}
nslookコマンドを、自宅のネットワークおよび学内ネットワークから
任意のサイトへ使用し、結果を表にまとめる。

\section{結果}
以下の表1,2は自宅と学内それぞれの結果を表にまとめたものである。\\
Google,YouTube,Twitterの3つは結果が異なったが、残りのものは全て
同じ結果となった。

\begin{table}[h]
  \centering
  \caption{学内ネットワーク}
  \begin{tabular}{|c|c|c|c|}
  \hline
  コンピュータの種類               & ホスト名(本名)              & IPアドレス          & ホスト名(別名) \\ \hline
  立命館大学 webサイト            & www.ritsumei.ac.jp    & 133.19.170.14   & なし       \\ \hline
  Apple                   & apple.com             & 17.253.144.10   & なし       \\ \hline
  Google                  & google.com            & 142.250.206.238 & なし       \\ \hline
  YouTube                 & youtbe.com            & 172.217.161.238 & なし       \\ \hline
  Twitter                 & twitter.com           & 104.244.42.193  & なし       \\ \hline
  Stone Island 公式サイト      & stoneisland.com       & 54.72.108.69    & なし       \\ \hline
  COMME des GARÇONS 公式サイト & comme-des-garcons.com & 205.186.183.189 & なし       \\ \hline
  \end{tabular}
  \end{table}

  \begin{table}[h]
    \centering
    \caption{自宅のネットワーク}
    \begin{tabular}{|c|c|c|c|}
    \hline
    コンピュータの種類               & ホスト名(本名)              & IPアドレス          & ホスト名(別名) \\ \hline
    立命館大学 webサイト            & www.ritsumei.ac.jp    & 133.19.170.14   & なし       \\ \hline
    Apple                   & apple.com             & 17.253.144.10   & なし       \\ \hline
    Google                  & google.com            & 142.250.207.110 & なし       \\ \hline
    YouTube                 & youtbe.com            & 142.250.206.238 & なし       \\ \hline
    Twitter                 & twitter.com           & 104.244.42.65  & なし       \\ \hline
    Stone Island 公式サイト      & stoneisland.com       & 54.72.108.69    & なし       \\ \hline
    COMME des GARÇONS 公式サイト & comme-des-garcons.com & 205.186.183.189 & なし       \\ \hline
    \end{tabular}
    \end{table}

\end{document}

