\documentclass[dvipdfmx,autodetect-engine,titlepage]{jsarticle}
\usepackage[dvipdfm]{graphicx}
\usepackage{ascmac}
\usepackage{fancybox}
\usepackage{listings}
\usepackage{plistings}
\usepackage{itembkbx}
\usepackage{amsmath}
\usepackage{svg}
\usepackage{url}
\usepackage{graphics}
\usepackage{listings,jvlisting}
\usepackage{scalefnt}
\usepackage{multirow}

\textheight=23cm
\renewcommand{\figurename}{図}
\renewcommand{\tablename}{表}
\newenvironment{code}
{\vspace{0.5zw}\VerbatimEnvironment  
\begin{screen} 
\baselineskip=1.0\normalbaselineskip
 \begin{Verbatim}}
{\end{Verbatim}
\baselineskip=\normalbaselineskip
 \end{screen}\vspace{0.5zw}} 

\title{情報理工学部 SNコース 2回\\
セキュリティ・ネットワーク学実験2\\
NW実験2-6レポート}
\author{2600200443-6\\Yamashita Kyohei\\山下 恭平}
\date{November 1 2021}

\begin{document}

\maketitle

\section{概要}
webサイトを閲覧できないというトラブルに関して,自分の経験や,今回学んだコマン
ド・ツール類の利用などによる調査,原因,および解決方法を考える。

\section{webサイトを閲覧できない理由とその調査}
必要な質問、考えられる原因、解決方法を以下の表にまとめた。

\begin{table}[h]
  \centering
  \caption{まとめ表}
  \begin{tabular}{|l|l|l|}
  \hline
  1)追加質問                                                                      & 2)考えられる原因                                                                                & 3)解決方法                                                                                                 \\ \hline \hline
  \multirow{2}{*}{\begin{tabular}[c]{@{}l@{}}問題のサイト以外は\\ 閲覧可能か?\end{tabular}} & \begin{tabular}[c]{@{}l@{}}(YES)サイト側の原因\\ と考えられる。\end{tabular}                           & \begin{tabular}[c]{@{}l@{}}サイト管理者に問い合わせるか、\\ 復旧を待つ。\end{tabular}                                         \\ \cline{2-3} 
                                                                              & \begin{tabular}[c]{@{}l@{}}(NO)コンピュータ側の問題\\ だと考えられる。\end{tabular}                        & \begin{tabular}[c]{@{}l@{}}コンピュータがインターネットに\\ 接続されているか確認し、接続され\\ ている場合は、更なる調査が必要。\end{tabular}            \\ \hline
  \begin{tabular}[c]{@{}l@{}}nslookup「サイトのホ ス\\ ト名」を実行させる\end{tabular}        & \begin{tabular}[c]{@{}l@{}}IPアドレスが返却されなけれ ば,\\ DNS に問題が生じていると考え\\ られる。\end{tabular} & \begin{tabular}[c]{@{}l@{}}更なる調査が必要(特にDNS障害の\\ 要因について)\end{tabular}                                      \\ \hline
  \begin{tabular}[c]{@{}l@{}}ウイルスソフトなどを\\ 使用しているか?\end{tabular}               & \begin{tabular}[c]{@{}l@{}}ソフトによってサイトがブロック\\ されている可能性がある。\end{tabular}                   & \begin{tabular}[c]{@{}l@{}}アクセスするサイトが安全かどうかを\\ 検討する必要あり。アクセスしたい場合\\ は設定からホワイトリストなどに登録す\\ る。\end{tabular} \\ \hline
  \begin{tabular}[c]{@{}l@{}}アクセスしたとき限られた\\ LANに接続していなかったか?\end{tabular}      & \begin{tabular}[c]{@{}l@{}}学内LANなどからの限定的なアクセス\\ しか受け付けないサーバの可能性がある。\end{tabular}         & 必要なLANに接続する必要がある。                                                                                        \\ \hline
  \begin{tabular}[c]{@{}l@{}}最近そのサイトにアク セスし\\ たのはいつ?\end{tabular}             & \begin{tabular}[c]{@{}l@{}}かなり昔であれば、そのサイト自体が無く\\ なってしまっている可能性がある。\end{tabular}           & \begin{tabular}[c]{@{}l@{}}Google,Yahooなど任意の検索エンジンを\\ 用いて、再検索してみる。\end{tabular}                          \\ \hline
  \end{tabular}
  \end{table}


\end{document}

